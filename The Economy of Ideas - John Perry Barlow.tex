% Options for packages loaded elsewhere
\PassOptionsToPackage{unicode}{hyperref}
\PassOptionsToPackage{hyphens}{url}
%
\documentclass[
]{article}
\usepackage{lmodern}
\usepackage{amssymb,amsmath}
\usepackage{ifxetex,ifluatex}
\ifnum 0\ifxetex 1\fi\ifluatex 1\fi=0 % if pdftex
  \usepackage[T1]{fontenc}
  \usepackage[utf8]{inputenc}
  \usepackage{textcomp} % provide euro and other symbols
\else % if luatex or xetex
  \usepackage{unicode-math}
  \defaultfontfeatures{Scale=MatchLowercase}
  \defaultfontfeatures[\rmfamily]{Ligatures=TeX,Scale=1}
\fi
% Use upquote if available, for straight quotes in verbatim environments
\IfFileExists{upquote.sty}{\usepackage{upquote}}{}
\IfFileExists{microtype.sty}{% use microtype if available
  \usepackage[]{microtype}
  \UseMicrotypeSet[protrusion]{basicmath} % disable protrusion for tt fonts
}{}
\makeatletter
\@ifundefined{KOMAClassName}{% if non-KOMA class
  \IfFileExists{parskip.sty}{%
    \usepackage{parskip}
  }{% else
    \setlength{\parindent}{0pt}
    \setlength{\parskip}{6pt plus 2pt minus 1pt}}
}{% if KOMA class
  \KOMAoptions{parskip=half}}
\makeatother
\usepackage{xcolor}
\IfFileExists{xurl.sty}{\usepackage{xurl}}{} % add URL line breaks if available
\IfFileExists{bookmark.sty}{\usepackage{bookmark}}{\usepackage{hyperref}}
\hypersetup{
  hidelinks,
  pdfcreator={LaTeX via pandoc}}
\urlstyle{same} % disable monospaced font for URLs
\setlength{\emergencystretch}{3em} % prevent overfull lines
\providecommand{\tightlist}{%
  \setlength{\itemsep}{0pt}\setlength{\parskip}{0pt}}
\setcounter{secnumdepth}{-\maxdimen} % remove section numbering
\ifluatex
  \usepackage{selnolig}  % disable illegal ligatures
\fi

\author{}
\date{}

\begin{document}

\hypertarget{header-n186}{%
\section{\texorpdfstring{The Economy of Ideas \textbar{}
\emph{WIRED}}{The Economy of Ideas \textbar{} WIRED}}\label{header-n186}}

\hypertarget{header-n188}{%
\subsection{A framework for patents and copyrights in the Digital Age.
(Everything you know about intellectual property is
wrong.)}\label{header-n188}}

John Perry Barlow

March 1994

\begin{quote}
"If nature has made any one thing less susceptible than all others of
exclusive property, it is the action of the thinking power called an
idea, which an individual may exclusively possess as long as he keeps it
to himself; but the moment it is divulged, it forces itself into the
possession of everyone, and the receiver cannot dispossess himself of
it. Its peculiar character, too, is that no one possesses the less,
because every other possesses the whole of it. He who receives an idea
from me, receives instruction himself without lessening mine; as he who
lights his taper at mine, receives light without darkening me. That
ideas should freely spread from one to another over the globe, for the
moral and mutual instruction of man, and improvement of his condition,
seems to have been peculiarly and benevolently designed by nature, when
she made them, like fire, expansible over all space, without lessening
their density at any point, and like the air in which we breathe, move,
and have our physical being, incapable of confinement or exclusive
appropriation. Inventions then cannot, in nature, be a subject of
property." - Thomas Jefferson
\end{quote}

\tableofcontents

\begin{center}\rule{0.5\linewidth}{0.5pt}\end{center}

Throughout the time I've been groping around cyberspace, an immense,
unsolved conundrum has remained at the root of nearly every legal,
ethical, governmental, and social vexation to be found in the Virtual
World. I refer to the problem of digitized property. The enigma is this:
If our property can be infinitely reproduced and instantaneously
distributed all over the planet without cost, without our knowledge,
without its even leaving our possession, how can we protect it? How are
we going to get paid for the work we do with our minds? And, if we can't
get paid, what will assure the continued creation and distribution of
such work?

Since we don't have a solution to what is a profoundly new kind of
challenge, and are apparently unable to delay the galloping digitization
of everything not obstinately physical, we are sailing into the future
on a sinking ship.

This vessel, the accumulated canon of copyright and patent law, was
developed to convey forms and methods of expression entirely different
from the vaporous cargo it is now being asked to carry. It is leaking as
much from within as from without.

Legal efforts to keep the old boat floating are taking three forms: a
frenzy of deck chair rearrangement, stern warnings to the passengers
that if she goes down, they will face harsh criminal penalties, and
serene, glassy-eyed denial.

Intellectual property law cannot be patched, retrofitted, or expanded to
contain digitized expression any more than real estate law might be
revised to cover the allocation of broadcasting spectrum (which, in
fact, rather resembles what is being attempted here). We will need to
develop an entirely new set of methods as befits this entirely new set
of circumstances.

Most of the people who actually create soft property - the programmers,
hackers, and Net surfers - already know this. Unfortunately, neither the
companies they work for nor the lawyers these companies hire have enough
direct experience with nonmaterial goods to understand why they are so
problematic. They are proceeding as though the old laws can somehow be
made to work, either by grotesque expansion or by force. They are wrong.

The source of this conundrum is as simple as its solution is complex.
Digital technology is detaching information from the physical plane,
where property law of all sorts has always found definition.

Throughout the history of copyrights and patents, the proprietary
assertions of thinkers have been focused not on their ideas but on the
expression of those ideas. The ideas themselves, as well as facts about
the phenomena of the world, were considered to be the collective
property of humanity. One could claim franchise, in the case of
copyright, on the precise turn of phrase used to convey a particular
idea or the order in which facts were presented.

The point at which this franchise was imposed was that moment when the
"word became flesh" by departing the mind of its originator and entering
some physical object, whether book or widget. The subsequent arrival of
other commercial media besides books didn't alter the legal importance
of this moment. Law protected expression and, with few (and recent)
exceptions, to express was to make physical.

Protecting physical expression had the force of convenience on its side.
Copyright worked well because, Gutenberg notwithstanding, it was hard to
make a book. Furthermore, books froze their contents into a condition
which was as challenging to alter as it was to reproduce. Counterfeiting
and distributing counterfeit volumes were obvious and visible activities
- it was easy enough to catch somebody in the act of doing. Finally,
unlike unbounded words or images, books had material surfaces to which
one could attach copyright notices, publisher's marques, and price tags.

Mental-to-physical conversion was even more central to patent. A patent,
until recently, was either a description of the form into which
materials were to be rendered in the service of some purpose, or a
description of the process by which rendition occurred. In either case,
the conceptual heart of patent was the material result. If no purposeful
object could be rendered because of some material limitation, the patent
was rejected. Neither a Klein bottle nor a shovel made of silk could be
patented. It had to be a thing, and the thing had to work.

Thus, the rights of invention and authorship adhered to activities in
the physical world. One didn't get paid for ideas, but for the ability
to deliver them into reality. For all practical purposes, the value was
in the conveyance and not in the thought conveyed.

In other words, the bottle was protected, not the wine.

Now, as information enters cyberspace, the native home of Mind, these
bottles are vanishing. With the advent of digitization, it is now
possible to replace all previous information storage forms with one
metabottle: complex and highly liquid patterns of ones and zeros.

Even the physical/digital bottles to which we've become accustomed -
floppy disks, CD-ROMs, and other discrete, shrink-wrappable bit-packages
- will disappear as all computers jack-in to the global Net. While the
Internet may never include every CPU on the planet, it is more than
doubling every year and can be expected to become the principal medium
of information conveyance, and perhaps eventually, the only one.

Once that has happened, all the goods of the Information Age - all of
the expressions once contained in books or film strips or newsletters -
will exist either as pure thought or something very much like thought:
voltage conditions darting around the Net at the speed of light, in
conditions that one might behold in effect, as glowing pixels or
transmitted sounds, but never touch or claim to "own" in the old sense
of the word.

Some might argue that information will still require some physical
manifestation, such as its magnetic existence on the titanic hard disks
of distant servers, but these are bottles which have no macroscopically
discrete or personally meaningful form.

Some will also argue that we have been dealing with unbottled expression
since the advent of radio, and they would be right. But for most of the
history of broadcast, there was no convenient way to capture soft goods
from the electromagnetic ether and reproduce them with quality available
in commercial packages. Only recently has this changed, and little has
been done legally or technically to address the change.

Generally, the issue of consumer payment for broadcast products was
irrelevant. The consumers themselves were the product. Broadcast media
were supported either by the sale of the attention of their audience to
advertisers, by government assessing payment through taxes, or by the
whining mendicancy of annual donor drives.

All of the broadcast-support models are flawed. Support either by
advertisers or government has almost invariably tainted the purity of
the goods delivered. Besides, direct marketing is gradually killing the
advertiser-support model anyway.

Broadcast media gave us another payment method for a virtual product:
the royalties that broadcasters pay songwriters through such
organizations as ASCAP and BMI. But, as a member of ASCAP, I can assure
you this is not a model that we should emulate. The monitoring methods
are wildly approximate. There is no parallel system of accounting in the
revenue stream. It doesn't really work. Honest.

In any case, without our old methods, based on physically defining the
expression of ideas, and in the absence of successful new models for
nonphysical transaction, we simply don't know how to assure reliable
payment for mental works. To make matters worse, this comes at a time
when the human mind is replacing sunlight and mineral deposits as the
principal source of new wealth.

Furthermore, the increasing difficulty of enforcing existing copyright
and patent laws is already placing in peril the ultimate source of
intellectual property - the free exchange of ideas.

That is, when the primary articles of commerce in a society look so much
like speech as to be indistinguishable from it, and when the traditional
methods of protecting their ownership have become ineffectual,
attempting to fix the problem with broader and more vigorous enforcement
will inevitably threaten freedom of speech. The greatest constraint on
your future liberties may come not from government but from corporate
legal departments laboring to protect by force what can no longer be
protected by practical efficiency or general social consent.

Furthermore, when Jefferson and his fellow creatures of the
Enlightenment designed the system that became American copyright law,
their primary objective was assuring the widespread distribution of
thought, not profit. Profit was the fuel that would carry ideas into the
libraries and minds of their new republic. Libraries would purchase
books, thus rewarding the authors for their work in assembling ideas;
these ideas, otherwise "incapable of confinement," would then become
freely available to the public. But what is the role of libraries in the
absence of books? How does society now pay for the distribution of ideas
if not by charging for the ideas themselves?

Additionally complicating the matter is the fact that along with the
disappearance of the physical bottles in which intellectual property
protection has resided, digital technology is also erasing the legal
jurisdictions of the physical world and replacing them with the
unbounded and perhaps permanently lawless waves of cyberspace.

In cyberspace, no national or local boundaries contain the scene of a
crime and determine the method of its prosecution; worse, no clear
cultural agreements define what a crime might be. Unresolved and basic
differences between Western and Asian cultural assumptions about
intellectual property can only be exacerbated when many transactions are
taking place in both hemispheres and yet, somehow, in neither.

Even in the most local of digital conditions, jurisdiction and
responsibility are hard to assess. A group of music publishers filed
suit against CompuServe this fall because it allowed its users to upload
musical compositions into areas where other users might access them. But
since CompuServe cannot practically exercise much control over the flood
of bits that passes between its subscribers, it probably shouldn't be
held responsible for unlawfully "publishing" these works.

Notions of property, value, ownership, and the nature of wealth itself
are changing more fundamentally than at any time since the Sumerians
first poked cuneiform into wet clay and called it stored grain. Only a
very few people are aware of the enormity of this shift, and fewer of
them are lawyers or public officials.

Those who do see these changes must prepare responses for the legal and
social confusion that will erupt as efforts to protect new forms of
property with old methods become more obviously futile, and, as a
consequence, more adamant.

\hypertarget{header-n223}{%
\subsection{From Swords to Writs to Bits}\label{header-n223}}

Humanity now seems bent on creating a world economy primarily based on
goods that take no material form. In doing so, we may be eliminating any
predictable connection between creators and a fair reward for the
utility or pleasure others may find in their works.

Without that connection, and without a fundamental change in
consciousness to accommodate its loss, we are building our future on
furor, litigation, and institutionalized evasion of payment except in
response to raw force. We may return to the Bad Old Days of property.

Throughout the darker parts of human history, the possession and
distribution of property was a largely military matter. "Ownership" was
assured those with the nastiest tools, whether fists or armies, and the
most resolute will to use them. Property was the divine right of thugs.

By the turn of the First Millennium AD, the emergence of merchant
classes and landed gentry forced the development of ethical
understandings for the resolution of property disputes. In the Middle
Ages, enlightened rulers like England's Henry II began to codify this
unwritten "common law" into recorded canons. These laws were local,
which didn't matter much as they were primarily directed at real estate,
a form of property that is local by definition. And, as the name
implied, was very real.

This continued to be the case as long as the origin of wealth was
agricultural, but with that dawning of the Industrial Revolution,
humanity began to focus as much on means as ends. Tools acquired a new
social value and, thanks to their development, it became possible to
duplicate and distribute them in quantity.

To encourage their invention, copyright and patent law were developed in
most Western countries. These laws were devoted to the delicate task of
getting mental creations into the world where they could be used - and
could enter the minds of others - while assuring their inventors
compensation for the value of their use. And, as previously stated, the
systems of both law and practice which grew up around that task were
based on physical expression.

Since it is now possible to convey ideas from one mind to another
without ever making them physical, we are now claiming to own ideas
themselves and not merely their expression. And since it is likewise now
possible to create useful tools that never take physical form, we have
taken to patenting abstractions, sequences of virtual events, and
mathematical formulae - the most unreal estate imaginable.

In certain areas, this leaves rights of ownership in such an ambiguous
condition that property again adheres to those who can muster the
largest armies. The only difference is that this time the armies consist
of lawyers.

Threatening their opponents with the endless purgatory of litigation,
over which some might prefer death itself, they assert claim to any
thought which might have entered another cranium within the collective
body of the corporations they serve. They act as though these ideas
appeared in splendid detachment from all previous human thought. And
they pretend that thinking about a product is somehow as good as
manufacturing, distributing, and selling it.

What was previously considered a common human resource, distributed
among the minds and libraries of the world, as well as the phenomena of
nature herself, is now being fenced and deeded. It is as though a new
class of enterprise had arisen that claimed to own the air.

What is to be done? While there is a certain grim fun to be had in it,
dancing on the grave of copyright and patent will solve little,
especially when so few are willing to admit that the occupant of this
grave is even deceased, and so many are trying to uphold by force what
can no longer be upheld by popular consent.

The legalists, desperate over their slipping grip, are vigorously trying
to extend their reach. Indeed, the United States and other proponents of
GATT are making adherence to our moribund systems of intellectual
property protection a condition of membership in the marketplace of
nations. For example, China will be denied Most Favored Nation trading
status unless they agree to uphold a set of culturally alien principles
that are no longer even sensibly applicable in their country of origin.

In a more perfect world, we'd be wise to declare a moratorium on
litigation, legislation, and international treaties in this area until
we had a clearer sense of the terms and conditions of enterprise in
cyberspace. Ideally, laws ratify already developed social consensus.
They are less the Social Contract itself than a series of memoranda
expressing a collective intent that has emerged out of many millions of
human interactions.

Humans have not inhabited cyberspace long enough or in sufficient
diversity to have developed a Social Contract which conforms to the
strange new conditions of that world. Laws developed prior to consensus
usually favor the already established few who can get them passed and
not society as a whole.

To the extent that law and established social practice exists in this
area, they are already in dangerous disagreement. The laws regarding
unlicensed reproduction of commercial software are clear and stern...and
rarely observed. Software piracy laws are so practically unenforceable
and breaking them has become so socially acceptable that only a thin
minority appears compelled, either by fear or conscience, to obey them.
When I give speeches on this subject, I always ask how many people in
the audience can honestly claim to have no unauthorized software on
their hard disks. I've never seen more than 10 percent of the hands go
up.

Whenever there is such profound divergence between law and social
practice, it is not society that adapts. Against the swift tide of
custom, the software publishers' current practice of hanging a few
visible scapegoats is so obviously capricious as to only further
diminish respect for the law.

Part of the widespread disregard for commercial software copyrights
stems from a legislative failure to understand the conditions into which
it was inserted. To assume that systems of law based in the physical
world will serve in an environment as fundamentally different as
cyberspace is a folly for which everyone doing business in the future
will pay.

As I will soon discuss in detail, unbounded intellectual property is
very different from physical property and can no longer be protected as
though these differences did not exist. For example, if we continue to
assume that value is based on scarcity, as it is with regard to physical
objects, we will create laws that are precisely contrary to the nature
of information, which may, in many cases, increase in value with
distribution.

The large, legally risk-averse institutions most likely to play by the
old rules will suffer for their compliance. As more lawyers, guns, and
money are invested in either protecting their rights or subverting those
of their opponents, their ability to produce new technology will simply
grind to a halt as every move they make drives them deeper into a tar
pit of courtroom warfare.

Faith in law will not be an effective strategy for high-tech companies.
Law adapts by continuous increments and at a pace second only to
geology. Technology advances in lunging jerks, like the punctuation of
biological evolution grotesquely accelerated. Real-world conditions will
continue to change at a blinding pace, and the law will lag further
behind, more profoundly confused. This mismatch may prove impossible to
overcome.

Promising economies based on purely digital products will either be born
in a state of paralysis, as appears to be the case with multimedia, or
continue in a brave and willful refusal by their owners to play the
ownership game at all.

In the United States one can already see a parallel economy developing,
mostly among small, fast moving enterprises who protect their ideas by
getting into the marketplace quicker then their larger competitors who
base their protection on fear and litigation.

Perhaps those who are part of the problem will simply quarantine
themselves in court, while those who are part of the solution will
create a new society based, at first, on piracy and freebooting. It may
well be that when the current system of intellectual property law has
collapsed, as seems inevitable, that no new legal structure will arise
in its place.

But something will happen. After all, people do business. When a
currency becomes meaningless, business is done in barter. When societies
develop outside the law, they develop their own unwritten codes,
practices, and ethical systems. While technology may undo law,
technology offers methods for restoring creative rights.

\textbf{A Taxonomy of Information}

It seems to me that the most productive thing to do now is to look into
the true nature of what we're trying to protect. How much do we really
know about information and its natural behaviors?

What are the essential characteristics of unbounded creation? How does
it differ from previous forms of property? How many of our assumptions
about it have actually been about its containers rather than their
mysterious contents? What are its different species and how does each of
them lend itself to control? What technologies will be useful in
creating new virtual bottles to replace the old physical ones?

Of course, information is, by nature, intangible and hard to define.
Like other such deep phenomena as light or matter, it is a natural host
to paradox. It is most helpful to understand light as being both a
particle and a wave, an understanding of information may emerge in the
abstract congruence of its several different properties which might be
described by the following three statements:

Information is an activity.\\
Information is a life form.\\
Information is a relationship.

In the following section, I will examine each of these.

\hypertarget{header-n254}{%
\subsection{I. INFORMATION IS AN ACTIVITY}\label{header-n254}}

\hypertarget{header-n255}{%
\subsubsection{Information Is a Verb, Not a Noun.}\label{header-n255}}

Freed of its containers, information is obviously not a thing. In fact,
it is something that happens in the field of interaction between minds
or objects or other pieces of information.

Gregory Bateson, expanding on the information theory of Claude Shannon,
said, "Information is a difference which makes a difference." Thus,
information only really exists in the Delta. The making of that
difference is an activity within a relationship. Information is an
action which occupies time rather than a state of being which occupies
physical space, as is the case with hard goods. It is the pitch, not the
baseball, the dance, not the dancer.

\hypertarget{header-n258}{%
\subsubsection{Information Is Experienced, Not
Possessed.}\label{header-n258}}

Even when it has been encapsulated in some static form like a book or a
hard disk, information is still something that happens to you as you
mentally decompress it from its storage code. But, whether it's running
at gigabits per second or words per minute, the actual decoding is a
process that must be performed by and upon a mind, a process that must
take place in time.

There was a cartoon in the Bulletin of Atomic Scientists a few years ago
that illustrated this point beautifully. In the drawing, a holdup man
trains his gun on the sort of bespectacled fellow you'd figure might
have a lot of information stored in his head. "Quick," orders the
bandit, "give me all your ideas."

\textbf{Information Has to Move.}

Sharks are said to die of suffocation if they stop swimming, and the
same is nearly true of information. Information that isn't moving ceases
to exist as anything but potential...at least until it is allowed to
move again. For this reason, the practice of information hoarding,
common in bureaucracies, is an especially wrong-headed artifact of
physically based value systems.

\textbf{Information Is Conveyed by Propagation, Not Distribution.}

The way in which information spreads is also very different from the
distribution of physical goods. It moves more like something from nature
than from a factory. It can concatenate like falling dominos or grow in
the usual fractal lattice, like frost spreading on a window, but it
cannot be shipped around like widgets, except to the extent that it can
be contained in them. It doesn't simply move on; it leaves a trail
everywhere it's been.

The central economic distinction between information and physical
property is that information can be transferred without leaving the
possession of the original owner. If I sell you my horse, I can't ride
him after that. If I sell you what I know, we both know it.

\hypertarget{header-n266}{%
\subsection{II. INFORMATION IS A LIFE FORM}\label{header-n266}}

\hypertarget{header-n267}{%
\subsubsection{Information Wants to Be Free.}\label{header-n267}}

Stewart Brand is generally credited with this elegant statement of the
obvious, which recognizes both the natural desire of secrets to be told
and the fact that they might be capable of possessing something like a
"desire" in the first place.

English biologist and philosopher Richard Dawkins proposed the idea of
"memes," self-replicating patterns of information that propagate
themselves across the ecologies of mind, a pattern of reproduction much
like that of life forms.

I believe they are life forms in every respect but their freedom from
the carbon atom. They self-reproduce, they interact with their
surroundings and adapt to them, they mutate, they persist. They evolve
to fill the empty niches of their local environments, which are, in this
case the surrounding belief systems and cultures of their hosts, namely,
us.

Indeed, sociobiologists like Dawkins make a plausible case that
carbon-based life forms are information as well, that, as the chicken is
an egg's way of making another egg, the entire biological spectacle is
just the DNA molecule's means of copying out more information strings
exactly like itself.

\hypertarget{header-n272}{%
\subsubsection{Information Replicates into the Cracks of
Possibility.}\label{header-n272}}

Like DNA helices, ideas are relentless expansionists, always seeking new
opportunities for Lebensraum. And, as in carbon-based nature, the more
robust organisms are extremely adept at finding new places to live.
Thus, just as the common housefly has insinuated itself into practically
every ecosystem on the planet, so has the meme of "life after death"
found a niche in most minds, or psycho-ecologies.

The more universally resonant an idea or image or song , the more minds
it will enter and remain within. Trying to stop the spread of a really
robust piece of information is about as easy as keeping killer bees
south of the border.

\hypertarget{header-n275}{%
\subsubsection{Information Wants to Change.}\label{header-n275}}

If ideas and other interactive patterns of information are indeed life
forms, they can be expected to evolve constantly into forms which will
be more perfectly adapted to their surroundings. And, as we see, they
are doing this all the time.

But for a long time, our static media, whether carvings in stone, ink on
paper, or dye on celluloid, have strongly resisted the evolutionary
impulse, exalting as a consequence the author's ability to determine the
finished product. But, as in an oral tradition, digitized information
has no "final cut."

Digital information, unconstrained by packaging, is a continuing process
more like the metamorphosing tales of prehistory than anything that will
fit in shrink-wrap. From the Neolithic to Gutenberg (monks aside),
information was passed on, mouth to ear, changing with every retelling
(or resinging). The stories which once shaped our sense of the world
didn't have authoritative versions. They adapted to each culture in
which they found themselves being told.

Because there was never a moment when the story was frozen in print, the
so-called "moral" right of storytellers to own the tale was neither
protected nor recognized. The story simply passed through each of them
on its way to the next, where it would assume a different form. As we
return to continuous information, we can expect the importance of
authorship to diminish. Creative people may have to renew their
acquaintance with humility.

But our system of copyright makes no accommodation whatever for
expressions which don't become fixed at some point nor for cultural
expressions which lack a specific author or inventor.

Jazz improvisations, stand-up comedy routines, mime performances,
developing monologues, and unrecorded broadcast transmissions all lack
the Constitutional requirement of fixation as a "writing." Without being
fixed by a point of publication the liquid works of the future will all
look more like these continuously adapting and changing forms and will
therefore exist beyond the reach of copyright.

Copyright expert Pamela Samuelson tells of having attended a conference
last year convened around the fact that Western countries may legally
appropriate the music, designs, and biomedical lore of aboriginal people
without compensation to their tribes of origin since those tribes are
not an "author" or "inventors."

But soon most information will be generated collaboratively by the
cyber-tribal hunter-gatherers of cyberspace. Our arrogant legal
dismissal of the rights of "primitives" will be soon return to haunt us.

\hypertarget{header-n284}{%
\subsubsection{Information Is Perishable.}\label{header-n284}}

With the exception of the rare classic, most information is like farm
produce. Its quality degrades rapidly both over time and in distance
from the source of production. But even here, value is highly subjective
and conditional. Yesterday's papers are quite valuable to the historian.
In fact, the older they are, the more valuable they become. On the other
hand, a commodities broker might consider news of an event that occurred
more than an hour ago to have lost any relevance.

\hypertarget{header-n286}{%
\subsection{III. INFORMATION IS A RELATIONSHIP}\label{header-n286}}

\hypertarget{header-n287}{%
\subsubsection{Meaning Has Value and Is Unique to Each
Case.}\label{header-n287}}

In most cases, we assign value to information based on its
meaningfulness. The place where information dwells, the holy moment
where transmission becomes reception, is a region which has many
shifting characteristics and flavors depending on the relationship of
sender and receiver, the depth of their interactivity.

Each such relationship is unique. Even in cases where the sender is a
broadcast medium, and no response is returned, the receiver is hardly
passive. Receiving information is often as creative an act as generating
it.

The value of what is sent depends entirely on the extent to which each
individual receiver has the receptors - shared terminology, attention,
interest, language, paradigm - necessary to render what is received
meaningful.

Understanding is a critical element increasingly overlooked in the
effort to turn information into a commodity. Data may be any set of
facts, useful or not, intelligible or inscrutable, germane or
irrelevant. Computers can crank out new data all night long without
human help, and the results may be offered for sale as information. They
may or may not actually be so. Only a human being can recognize the
meaning that separates information from data.

In fact, information, in the economic sense of the word, consists of
data which have been passed through a particular human mind and found
meaningful within that mental context. One fella's information is all
just data to someone else. If you're an anthropologist, my detailed
charts of Tasaday kinship patterns might be critical information to you.
If you're a banker from Hong Kong, they might barely seem to be data.

\hypertarget{header-n294}{%
\subsubsection{Familiarity Has More Value than
Scarcity.}\label{header-n294}}

With physical goods, there is a direct correlation between scarcity and
value. Gold is more valuable than wheat, even though you can't eat it.
While this is not always the case, the situation with information is
often precisely the reverse. Most soft goods increase in value as they
become more common. Familiarity is an important asset in the world of
information. It may often be true that the best way to raise demand for
your product is to give it away.

While this has not always worked with shareware, it could be argued that
there is a connection between the extent to which commercial software is
pirated and the amount which gets sold. Broadly pirated software, such
as Lotus 1-2-3 or WordPerfect, becomes a standard and benefits from Law
of Increasing Returns based on familiarity.

In regard to my own soft product, rock 'n' roll songs, there is no
question that the band I write them for, the Grateful Dead, has
increased its popularity enormously by giving them away. We have been
letting people tape our concerts since the early seventies, but instead
of reducing the demand for our product, we are now the largest concert
draw in America, a fact that is at least in part attributable to the
popularity generated by those tapes.

True, I don't get any royalties on the millions of copies of my songs
which have been extracted from concerts, but I see no reason to
complain. The fact is, no one but the Grateful Dead can perform a
Grateful Dead song, so if you want the experience and not its thin
projection, you have to buy a ticket from us. In other words, our
intellectual property protection derives from our being the only
real-time source of it.

\hypertarget{header-n299}{%
\subsubsection{Exclusivity Has Value.}\label{header-n299}}

The problem with a model that turns the physical scarcity/value ratio on
its head is that sometimes the value of information is very much based
on its scarcity. Exclusive possession of certain facts makes them more
useful. If everyone knows about conditions which might drive a stock
price up, the information is valueless.

But again, the critical factor is usually time. It doesn't matter if
this kind of information eventually becomes ubiquitous. What matters is
being among the first who possess it and act on it. While potent secrets
usually don't stay secret, they may remain so long enough to advance the
cause of their original holders.

\hypertarget{header-n302}{%
\subsubsection{Point of View and Authority Have
Value.}\label{header-n302}}

In a world of floating realities and contradictory maps, rewards will
accrue to those commentators whose maps seem to fit their territory
snugly, based on their ability to yield predictable results for those
who use them.

In aesthetic information, whether poetry or rock 'n' roll, people are
willing to buy the new product of an artist, sight-unseen, based on
their having been delivered a pleasurable experience by previous work.

Reality is an edit. People are willing to pay for the authority of those
editors whose point of view seems to fit best. And again, point of view
is an asset which cannot be stolen or duplicated. No one sees the world
as Esther Dyson does, and the handsome fee she charges for her
newsletter is actually payment for the privilege of looking at the world
through her unique eyes.

\hypertarget{header-n306}{%
\subsubsection{Time Replaces Space.}\label{header-n306}}

In the physical world, value depends heavily on possession or proximity
in space. One owns the material that falls inside certain dimensional
boundaries. The ability to act directly, exclusively, and as one wishes
upon what falls inside those boundaries is the principal right of
ownership. The relationship between value and scarcity is a limitation
in space.

In the virtual world, proximity in time is a value determinant. An
informational product is generally more valuable the closer purchaser
can place themselves to the moment of its expression, a limitation in
time. Many kinds of information degrade rapidly with either time or
reproduction. Relevance fades as the territory they map changes. Noise
is introduced and bandwidth lost with passage away from the point where
the information is first produced.

Thus, listening to a Grateful Dead tape is hardly the same experience as
attending a Grateful Dead concert. The closer one can get to the
headwaters of an informational stream, the better one's chances of
finding an accurate picture of reality in it. In an era of easy
reproduction, the informational abstractions of popular experiences will
propagate out from their source moments to reach anyone who's
interested. But it's easy enough to restrict the real experience of the
desirable event, whether knock-out punch or guitar lick, to those
willing to pay for being there.

\hypertarget{header-n321}{%
\subsubsection{The Protection of Execution}\label{header-n321}}

In the hick town I come from, they don't give you much credit for just
having ideas. You are judged by what you can make of them. As things
continue to speed up, I think we see that execution is the best
protection for those designs which become physical products. Or, as
Steve Jobs once put it, "Real artists ship." The big winner is usually
the one who gets to the market first (and with enough organizational
force to keep the lead).

But, as we become fixated upon information commerce, many of us seem to
think that originality alone is sufficient to convey value, deserving,
with the right legal assurances, of a steady wage. In fact, the best way
to protect intellectual property is to act on it. It's not enough to
invent and patent; one has to innovate as well. Someone claims to have
patented the microprocessor before Intel. Maybe so. If he'd actually
started shipping microprocessors before Intel, his claim would seem far
less spurious.

\hypertarget{header-n324}{%
\subsubsection{Information as Its Own Reward}\label{header-n324}}

It is now a commonplace to say that money is information. With the
exception of Krugerrands, crumpled cab fare, and the contents of those
suitcases that drug lords are reputed to carry, most of the money in the
informatized world is in ones and zeros. The global money supply sloshes
around the Net, as fluid as weather. It is also obvious, that
information has become as fundamental to the creation of modern wealth
as land and sunlight once were.

What is less obvious is the extent to which information is acquiring
intrinsic value, not as a means to acquisition but as the object to be
acquired. I suppose this has always been less explicitly the case. In
politics and academia, potency and information have always been closely
related.

However, as we increasingly buy information with money, we begin to see
that buying information with other information is simple economic
exchange without the necessity of converting the product into and out of
currency. This is somewhat challenging for those who like clean
accounting, since, information theory aside, informational exchange
rates are too squishy to quantify to the decimal point.

Nevertheless, most of what a middle-class American purchases has little
to do with survival. We buy beauty, prestige, experience, education, and
all the obscure pleasures of owning. Many of these things can not only
be expressed in nonmaterial terms, they can be acquired by nonmaterial
means.

And then there are the inexplicable pleasures of information itself, the
joys of learning, knowing, and teaching; the strange good feeling of
information coming into and out of oneself. Playing with ideas is a
recreation which people are willing to pay a lot for, given the market
for books and elective seminars. We'd likely spend even more money for
such pleasures if we didn't have so many opportunities to pay for ideas
with other ideas. This explains much of the collective "volunteer" work
which fills the archives, newsgroups, and databases of the Internet. Its
denizens are not working for "nothing," as is widely believed. Rather
they are getting paid in something besides money. It is an economy which
consists almost entirely of information.

This may become the dominant form of human trade, and if we persist in
modeling economics on a strictly monetary basis, we may be gravely
misled.

\hypertarget{header-n337}{%
\subsubsection{Getting Paid in Cyberspace}\label{header-n337}}

How all the foregoing relates to solutions to the crisis in intellectual
property is something I've barely started to wrap my mind around. It's
fairly paradigm warping to look at information through fresh eyes - to
see how very little it is like pig iron or pork bellies, and to imagine
the tottering travesties of case law we will stack up if we go on
legally treating it as though it were.

As I've said, I believe these towers of outmoded boilerplate will be a
smoking heap sometime in the next decade, and we mind miners will have
no choice but to cast our lot with new systems that work.

I'm not really so gloomy about our prospects as readers of this jeremiad
so far might conclude. Solutions will emerge. Nature abhors a vacuum and
so does commerce.

Indeed, one of the aspects of the electronic frontier which I have
always found most appealing - and the reason Mitch Kapor and I used that
phrase in naming our foundation - is the degree to which it resembles
the 19th-century American West in its natural preference for social
devices that emerge from its conditions rather than those that are
imposed from the outside.

Until the West was fully settled and "civilized" in this century, order
was established according to an unwritten Code of the West, which had
the fluidity of common law rather than the rigidity of statutes. Ethics
were more important than rules. Understandings were preferred over laws,
which were, in any event, largely unenforceable.

I believe that law, as we understand it, was developed to protect the
interests which arose in the two economic "waves" which Alvin Toffler
accurately identified in The Third Wave. The First Wave was
agriculturally based and required law to order ownership of the
principal source of production, land. In the Second Wave, manufacturing
became the economic mainspring, and the structure of modern law grew
around the centralized institutions that needed protection for their
reserves of capital, labor, and hardware.

Both of these economic systems required stability. Their laws were
designed to resist change and to assure some equability of distribution
within a fairly static social framework. The empty niches had to be
constrained to preserve the predictability necessary to either land
stewardship or capital formation.

In the Third Wave we have now entered, information to a large extent
replaces land, capital, and hardware, and information is most at home in
a much more fluid and adaptable environment. The Third Wave is likely to
bring a fundamental shift in the purposes and methods of law which will
affect far more than simply those statutes which govern intellectual
property.

The "terrain" itself - the architecture of the Net - may come to serve
many of the purposes which could only be maintained in the past by legal
imposition. For example, it may be unnecessary to constitutionally
assure freedom of expression in an environment which, in the words of my
fellow EFF co-founder John Gilmore, "treats censorship as a malfunction"
and reroutes proscribed ideas around it.

Similar natural balancing mechanisms may arise to smooth over the social
discontinuities which previously required legal intercession to set
right. On the Net, these differences are more likely to be spanned by a
continuous spectrum that connects as much as it separates.

And, despite their fierce grip on the old legal structure, companies
that trade in information are likely to find that their increasing
inability to deal sensibly with technological issues will not be
remedied in the courts, which won't be capable of producing verdicts
predictable enough to be supportive of long-term enterprise. Every
litigation will become like a game of Russian roulette, depending on the
depth of the presiding judge's clue-impairment.

Uncodified or adaptive "law," while as "fast, loose, and out of control"
as other emergent forms, is probably more likely to yield something like
justice at this point. In fact, one can already see in development new
practices to suit the conditions of virtual commerce. The life forms of
information are evolving methods to protect their continued
reproduction.

For example, while all the tiny print on a commercial diskette envelope
punctiliously requires a great deal of those who would open it, few who
read those provisos follow them to the letter. And yet, the software
business remains a very healthy sector of the American economy.

Why is this? Because people seem to eventually buy the software they
really use. Once a program becomes central to your work, you want the
latest version of it, the best support, the actual manuals, all
privileges attached to ownership. Such practical considerations will, in
the absence of working law, become more and more important in getting
paid for what might easily be obtained for nothing.

I do think that some software is being purchased in the service of
ethics or the abstract awareness that the failure to buy it will result
in its not being produced any longer, but I'm going to leave those
motivators aside. While I believe that the failure of law will almost
certainly result in a compensating re-emergence of ethics as the
ordering template of society, this is a belief I don't have room to
support here.

Instead, I think that, as in the case cited above, compensation for soft
products will be driven primarily by practical considerations, all of
them consistent with the true properties of digital information, where
the value lies in it, and how it can be both manipulated and protected
by technology.

While the conundrum remains a conundrum, I can begin to see the
directions from which solutions may emerge, based in part on broadening
those practical solutions which are already in practice.

\hypertarget{header-n359}{%
\subsubsection{Relationship and Its Tools}\label{header-n359}}

I believe one idea is central to understanding liquid commerce:
Information economics, in the absence of objects, will be based more on
relationship than possession.

One existing model for the future conveyance of intellectual property is
real-time performance, a medium currently used only in theater, music,
lectures, stand-up comedy, and pedagogy. I believe the concept of
performance will expand to include most of the information economy, from
multicasted soap operas to stock analysis. In these instances,
commercial exchange will be more like ticket sales to a continuous show
than the purchase of discrete bundles of that which is being shown.

The other existing, model, of course, is service. The entire
professional class - doctors, lawyers, consultants, architects, and so
on - are already being paid directly for their intellectual property.
Who needs copyright when you're on a retainer?

In fact, until the late 18th century this model was applied to much of
what is now copyrighted. Before the industrialization of creation,
writers, composers, artists, and the like produced their products in the
private service of patrons. Without objects to distribute in a mass
market, creative people will return to a condition somewhat like this,
except that they will serve many patrons, rather than one.

We can already see the emergence of companies which base their existence
on supporting and enhancing the soft property they create rather than
selling it by the shrink-wrapped piece or embedding it in widgets.

Trip Hawkins's new company for creating and licensing multimedia tools,
3DO, is an example of what I'm talking about. 3DO doesn't intend to
produce any commercial software or consumer devices. Instead, it will
act as a kind of private standards setting body, mediating among
software and device creators who will be their licensees. It will
provide a point of commonality for relationships between a broad
spectrum of entities.

In any case, whether you think of yourself as a service provider or a
performer, the future protection of your intellectual property will
depend on your ability to control your relationship to the market - a
relationship which will most likely live and grow over a period of time.

The value of that relationship will reside in the quality of
performance, the uniqueness of your point of view, the validity of your
expertise, its relevance to your market, and, underlying everything, the
ability of that market to access your creative services swiftly,
conveniently, and interactively.

\hypertarget{header-n373}{%
\subsubsection{Interaction and Protection}\label{header-n373}}

Direct interaction will provide a lot of intellectual property
protection in the future, and, indeed, already has. No one knows how
many software pirates have bought legitimate copies of a program after
calling its publisher for technical support and offering some proof of
purchase, but I would guess the number is very high.

The same kind of controls will be applicable to "question and answer"
relationships between authorities (or artists) and those who seek their
expertise. Newsletters, magazines, and books will be supplemented by the
ability of their subscribers to ask direct questions of authors.

Interactivity will be a billable commodity even in the absence of
authorship. As people move into the Net and increasingly get their
information directly from its point of production, unfiltered by
centralized media, they will attempt to develop the same interactive
ability to probe reality that only experience has provided them in the
past. Live access to these distant "eyes and ears" will be much easier
to cordon than access to static bundles of stored but easily
reproducible information.

In most cases, control will be based on restricting access to the
freshest, highest bandwidth information. It will be a matter of defining
the ticket, the venue, the performer, and the identity of the ticket
holder, definitions which I believe will take their forms from
technology, not law. In most cases, the defining technology will be
cryptography.

\hypertarget{header-n378}{%
\subsubsection{Crypto Bottling}\label{header-n378}}

Cryptography, as I've said perhaps too many times, is the "material"
from which the walls, boundaries - and bottles - of cyberspace will be
fashioned.

Of course there are problems with cryptography or any other purely
technical method of property protection. It has always appeared to me
that the more security you hide your goods behind, the more likely you
are to turn your sanctuary into a target. Having come from a place where
people leave their keys in their cars and don't even have keys to their
houses, I remain convinced that the best obstacle to crime is a society
with its ethics intact.

While I admit that this is not the kind of society most of us live in, I
also believe that a social over reliance on protection by barricades
rather than conscience will eventually wither the latter by turning
intrusion and theft into a sport, rather than a crime. This is already
occurring in the digital domain as is evident in the activities of
computer crackers.

Furthermore, I would argue that initial efforts to protect digital
copyright by copy protection contributed to the current condition in
which most otherwise ethical computer users seem morally untroubled by
their possession of pirated software.

Instead of cultivating among the newly computerized a sense of respect
for the work of their fellows, early reliance on copy protection led to
the subliminal notion that cracking into a software package somehow
"earned" one the right to use it. Limited not by conscience but by
technical skill, many soon felt free to do whatever they could get away
with. This will continue to be a potential liability of the encryption
of digitized commerce.

Furthermore, it's cautionary to remember that copy protection was
rejected by the market in most areas. Many of the upcoming efforts to
use cryptography-based protection schemes will probably suffer the same
fate. People are not going to tolerate much that makes computers harder
to use than they already are without any benefit to the user.

Nevertheless, encryption has already demonstrated a certain blunt
utility. New subscriptions to various commercial satellite TV services
skyrocketed recently after their deployment of more robust encryption of
their feeds. This, despite a booming backwoods trade in black decoder
chips, conducted by folks who'd look more at home running moonshine than
cracking code.

Another obvious problem with encryption as a global solution is that
once something has been unscrambled by a legitimate licensee, it may be
available to massive reproduction.

In some instances, reproduction following decryption may not be a
problem. Many soft products degrade sharply in value with time. It may
be that the only real interest in such products will be among those who
have purchased the keys to immediacy.

Furthermore, as software becomes more modular and distribution moves
online, it will begin to metamorphose in direct interaction with its
user base. Discontinuous upgrades will smooth into a constant process of
incremental improvement and adaptation, some of it manmade and some of
it arising through genetic algorithms. Pirated copies of software may
become too static to have much value to anyone.

Even in cases such as images, where the information is expected to
remain fixed, the unencrypted file could still be interwoven with code
which could continue to protect it by a wide variety of means.

In most of the schemes I can project, the file would be "alive" with
permanently embedded software that could "sense" the surrounding
conditions and interact with them. For example, it might contain code
that could detect the process of duplication and cause it to
self-destruct.

Other methods might give the file the ability to "phone home" through
the Net to its original owner. The continued integrity of some files
might require periodic "feeding" with digital cash from their host,
which they would then relay back to their authors.

Of course files that possess the independent ability to communicate
upstream sound uncomfortably like the Morris Internet Worm. "Live" files
do have a certain viral quality. And serious privacy issues would arise
if everyone's computer were packed with digital spies.

The point is that cryptography will enable protection technologies that
will develop rapidly in the obsessive competition that has always
existed between lock-makers and lock-breakers.

But cryptography will not be used simply for making locks. It is also at
the heart of both digital signatures and the aforementioned digital
cash, both of which I believe will be central to the future protection
of intellectual property.

I believe that the generally acknowledged failure of the shareware model
in software had less to do with dishonesty than with the simple
inconvenience of paying for shareware. If the payment process can be
automated, as digital cash and signature will make possible, I believe
that soft product creators will reap a much higher return from the bread
they cast upon the waters of cyberspace.

Moreover, they will be spared much of the overhead presently attached to
the marketing, manufacture, sales, and distribution of information
products, whether those products are computer programs, books, CDs, or
motion pictures. This will reduce prices and further increase the
likelihood of noncompulsory payment.

But of course there is a fundamental problem with a system that
requires, through technology, payment for every access to a particular
expression. It defeats the original Jeffersonian purpose of seeing that
ideas were available to everyone regardless of their economic station. I
am not comfortable with a model that will restrict inquiry to the
wealthy.

\hypertarget{header-n407}{%
\subsubsection{An Economy of Verbs}\label{header-n407}}

The future forms and protections of intellectual property are densely
obscured at this entrance to the Virtual Age. Nevertheless, I can make
(or reiterate) a few flat statements that I earnestly believe won't look
too silly in 50 years.

\begin{itemize}
\item
  In the absence of the old containers, almost everything we think we
  know about intellectual property is wrong. We're going to have to
  unlearn it. We're going to have to look at information as though we'd
  never seen the stuff before.
\item
  The protections that we will develop will rely far more on ethics and
  technology than on law.
\item
  Encryption will be the technical basis for most intellectual property
  protection. (And should, for many reasons, be made more widely
  available.)
\item
  The economy of the future will be based on relationship rather than
  possession. It will be continuous rather than sequential.
\item
  And finally, in the years to come, most human exchange will be virtual
  rather than physical, consisting not of stuff but the stuff of which
  dreams are made. Our future business will be conducted in a world made
  more of verbs than nouns.
\end{itemize}

\end{document}
